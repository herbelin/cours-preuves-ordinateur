\documentclass{article}
\usepackage{fontspec,mathpartir,amsmath,minted}
\usepackage[a4paper]{geometry}
\usepackage[francais]{babel}
\DeclareMathOperator{\efq}{efq}
\DeclareMathOperator{\case}{case}
\DeclareMathOperator{\of}{of}
\DeclareMathOperator{\dest}{dest}
\DeclareMathOperator{\as}{as}
\DeclareMathOperator{\cin}{in}
\usepackage{enumitem}
\begin{document}
\title{Arithmétique}
\author{Preuves assistées par ordinateur -- TP 2 -- 7 mars 2025}
\date{Année scolaire 2024--2025}
\maketitle

\section*{Exercice 1 : Quelques résultats d'arithmétique}

On se propose de redémontrer quelques résultats d'arithmétiques déjà démontrés dans la bibliothèque standard de Coq (on indique entre parenthères le nom des énoncés correspondants).

\begin{enumerate}
\item Démontrer que l'addition est commutative (\mintinline{coq}{Nat.add_comm}) et associative (\mintinline{coq}{Nat.add_assoc}) et admet 0 pour neutre (\mintinline{coq}{plus_O_n}).
\item Démontrer que la multiplication est distributive par rapport à l'addition (\mintinline{coq}{Nat.mul_add_distr_l}).
\item Démontrer que la multiplication est commutative (\mintinline{coq}{Nat.mul_comm}) et associative (\mintinline{coq}{Nat.mul_assoc}) et admet 1 pour neutre (\mintinline{coq}{Nat.mul_1_l}) et 0 pour élément absorbant (\mintinline{coq}{Nat.mul_0_l}).
\end{enumerate}

\section*{Exercice 2 : Compilation vers une machine à pile}
On considère la définition inductive suivante pour représenter les expressions arithmétiques.

\begin{minted}{coq}
Inductive expr :=
| Constant (_ : nat)
| Plus (_ _ : expr)
| Times (_ _ : expr).
\end{minted}

\begin{enumerate}

\item Écrire une fonction \mintinline{coq}{eval} qui calcule la valeur de l'expression.
\end{enumerate}

On considère maintenant le langage à pile constitué des listes des commandes suivantes.

\begin{minted}{coq}
Inductive command :=
| Push (_ : nat)
| Add
| Mul.
\end{minted}

\begin{enumerate}[resume]
\item Écrire une fonction \mintinline{coq}{eval_stack} qui calcule la valeur d'une liste de commandes étant donnéee une pile initiale. On utilisera un type \mintinline{coq}{option} et la valeur \mintinline{coq}{None} en cas d'erreur de pile.

\item Écrire une fonction \mintinline{coq}{compile} qui traduit une expression et une liste de commandes et démontrer que la valeur évaluée est préservée.
\end{enumerate}



\end{document}
%%% Local Variables:
%%% coding: utf-8
%%% mode: latex
%%% TeX-engine: luatex
%%% TeX-command-extra-options: "-shell-escape"
%%% End:
