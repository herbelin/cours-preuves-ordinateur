\documentclass{article}
\usepackage{fontspec,mathpartir,amsmath,minted}
\usepackage[a4paper]{geometry}
\usepackage[francais]{babel}
\DeclareMathOperator{\efq}{efq}
\DeclareMathOperator{\case}{case}
\DeclareMathOperator{\of}{of}
\DeclareMathOperator{\dest}{dest}
\DeclareMathOperator{\as}{as}
\DeclareMathOperator{\cin}{in}
\usepackage{enumitem}
\begin{document}
\title{Autour de l'ensemble \mintinline{coq}{Prop} des valeurs de vérité}
\author{Preuves assistées par ordinateur -- TP 3 -- 21 mars 2025}
\date{Année scolaire 2024--2025}
\maketitle

\section*{Exercice 1 : le paradoxe du buveur et autres versions faibles du tiers exclu}

Le paradoxe du buveur est généralement présenté ainsi : « dans toute pièce non vide, il existe une personne qui, dès que quelqu'un boit, alors elle boit. ».

On peut l'énoncer en Rocq de la manière suivante :

\begin{minted}{coq}
Definition DrinkerParadox :=
  forall (A: Type) (P: A -> Prop),
    A -> exists x, (exists y, P y) -> P x.
\end{minted}

\begin{enumerate}
\item Montrer que le tiers exclu implique le paradoxe du buveur.
\end{enumerate}

On appelle principe d'indépendance des prémisses générales le fait que si une prémisse \(Q\) implique \(\exists x.P(x)\), alors le choix de \(x\) ne dépend pas de \(Q\) : autrement dit, il existe \(x\) tel que \(Q\) implique \(P(x)\).

\begin{minted}{coq}
Definition IndependenceOfGeneralPremises :=
  forall (A:Type) (P:A -> Prop) (Q:Prop),
    A -> (Q -> exists x, P x) -> exists x, Q -> P x.
\end{minted}

\begin{enumerate}[resume]
\item Montrer que le paradoxe du buveur est équivalent au principe d'indépendance des prémisses générales.
\item Montrer que le paradoxe du buveur implique la loi de Gödel-Dummet : pour toutes propositions \(A\) et \(B\), soit \(A\) implique \(B\), soit \(B\) implique \(A\).
\item Montrer que la loi de Gödel-Dummet est équivalente à la distributivité à droite de l'implication sur la disjonction : pour toutes propositions \(A\), \(B\) et \(C\), si \(C \Rightarrow A \vee B\), alors \((C \Rightarrow A) \vee (C \Rightarrow B)\).
\item Montrer que la loi de Gödel-Dummet implique le tiers exclu faible : pour toute proposition \(A\), on a soit \(\neg \neg A\), soit \(\neg A\).
\item Montrer que le tiers exclu faible est équivalent à la loi de de Morgan suivante : pour toute proposition \(A\) et \(B\), si \(\neg (A \wedge B)\), alors \(\neg A \vee \neg B\).
\end{enumerate}

La contraposée du paradoxe du buveur peut s'énoncer ainsi : « dans toute pièce non vide, il existe une personne telle que si cette personne boit, alors tout le monde dans la pièce boit. ».

\begin{minted}{coq}
Definition DrinkerParadox' :=
  forall (A: Type) (P: A -> Prop),
    A -> exists x, P x -> (forall y, P y).
\end{minted}

\begin{enumerate}[resume]
\item Montrer cette version du paradoxe du buveur en supposant le tiers exclu.
\end{enumerate}

\section*{Exercice 2 : complétude propositionnelle,  extensionalité propositionnelle et tiers exclu}

La complétude (ou dégénérescence) propositionnelle énonce que les seules valeurs de vérité sont \mintinline{coq}{True} et \mintinline{coq}{False}.

\begin{minted}{coq}
Definition prop_degeneracy := forall A:Prop, A = True \/ A = False.
\end{minted}

L'extensionalité propositionnelle énonce que deux propositions équivalentes sont égales.

\begin{minted}{coq}
Definition prop_extensionality := forall A B:Prop, (A <-> B) -> A = B.
\end{minted}

\begin{enumerate}
\item Montrer que la complétude propositionnelle implique le tiers exclu et l'extensionalité propositionnelle.
\item Montrer réciproquement que la conjonction du tiers exclu et de l'extensionalité propositionnelle implique la complétude propositionnelle.
\end{enumerate}

On suppose l'extensionalité propositionnelle.

\begin{enumerate}[resume]
\item Montrer que, pour toute proposition \(A\), si \(A\) est vraie, alors \((A \rightarrow A) = A\).
\item Montrer que, pour toute proposition \(A\), si \(A\) est vraie, alors il existe deux fonctions \(f : A \rightarrow (A \rightarrow A)\)
et \(g : (A \rightarrow A) \rightarrow A\) telles que \(f \circ g = \mathrm{id}_{A \rightarrow A}\).
\item Montrer que, pour toute proposition \(A\), si \(A\) est vraie, alors il existe un opérateur de point fixe pour \(A\), c'est-à-dire une
  fonction \(F : (A \rightarrow A) \rightarrow A\) telle que pour toute fonction \(f: A \rightarrow A\), \(F f\) est un point fixe de \(f\),
  c'est-à-dire \(f (F f) = F f\).
\end{enumerate}

\section*{Exercice 3 : théorème d'involution de Higgs}

Montrer le théorème d'involution de Higgs.
\begin{minted}{coq}
Theorem Higgs_involution (f : Prop -> Prop) :
   (forall P Q, (P <-> Q) <-> (f P <-> f Q)) -> (forall P, P <-> f (f P)).
\end{minted}

Cette version du théorème vaut en théorie des types (sans axiome supplémentaire !).
Si on suppose en plus l'extensionalité propositionnelle,
l'énoncé peut être reformulé de la manière suivante : si \(f\) est injective, alors en fait \(f \circ f = \mathrm{id}_{\mintinline{coq}{Prop}}\).

\end{document}
%%% Local Variables:
%%% coding: utf-8
%%% mode: latex
%%% TeX-engine: luatex
%%% TeX-command-extra-options: "-shell-escape"
%%% End:
