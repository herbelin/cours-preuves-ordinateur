\documentclass{article}
\usepackage{fontspec,mathpartir,amsmath,minted}
\usepackage[a4paper]{geometry}
\usepackage[francais]{babel}
\DeclareMathOperator{\efq}{efq}
\DeclareMathOperator{\case}{case}
\DeclareMathOperator{\of}{of}
\DeclareMathOperator{\dest}{dest}
\DeclareMathOperator{\as}{as}
\DeclareMathOperator{\cin}{in}
\begin{document}
\title{Logique constructive et types dépendants}
\author{Preuves assistées par ordinateur -- TP 1 -- 31 janvier 2025}
\date{Année scolaire 2024--2025}
\maketitle

\section*{Exercice 1 : Un peu de théorie constructive des ensembles}

On se donne deux ensembles \mintinline{coq}{A} et \mintinline{coq}{B},
qu'on définit une fois pour toutes les définitions qui vont suivre.

\begin{minted}{coq}
Parameters A B : Set.
\end{minted}

\begin{enumerate}
\item Définir une propriété \mintinline{coq}{injective: (A -> B) -> Prop} sur
les fonctions \mintinline{coq}{A -> B} telle qu'on ait
\mintinline{coq}{injective f} lorsque \mintinline{coq}{f} est injective
(\emph{c.-à.-d.}, pour tous \(x\) et \(y\), si \(f x = f y\), alors \(x = y\)).
\item Montrer que \mintinline{coq}{f} est injective si et seulement si
\mintinline{coq}{f} est simplifiable à gauche
(\emph{c.-à.-d.}, pour toutes fonctions \(g\) et \(g'\), si \(f \circ g = f \circ g'\),
alors \(g = g'\)).
\item Définir une propriété \mintinline{coq}{surjective: (A -> B) -> Prop} sur
les fonctions \mintinline{coq}{A -> B} telle qu'on ait
\mintinline{coq}{surjective f} lorsque \mintinline{coq}{f} est surjective
(\emph{c.-à.-d.}, pour tout \(y\), il existe \(x\), tel que \(f x = y\)).
\item En supposant le tiers exclu, montrer que \mintinline{coq}{f} est surjective si et seulement si
\mintinline{coq}{f} est simplifiable à droite
(\emph{c.-à.-d.}, pour toutes fonctions \(g\) et \(g'\), si \(g \circ f = g' \circ f\),
alors \(g = g'\)).

\textbf{Erratum :} On supposera aussi l'axiome d'extentionnalité propositionnelle :

\mintinline{coq}{forall p q : Prop, (p <-> q) -> p = q}

On peut noter que la conjonction de l'axiome d'extentionnalité propositionnelle et du tiers exclu est équivalente à l'axiome de dégénérescence propositionnelle (ou complétude propositionnelle) : \mintinline{coq}{forall p : Prop, p = True \/ p = False}. Ces axiomes sont définis dans le module \texttt{ClassicalFacts} de la bibliothèque standard.

\textbf{Indications :} Le tiers exclu permet de supposer par l'absurde 

\mintinline{coq}{~ (exists x : A, f x = y)}

On pourra poser les définitions suivantes :
\begin{minted}{coq}
set (g := fun _: B => False).
set (g' := fun y' => y = y').
\end{minted}

On peut alors montrer facilement que \mintinline{coq}{g y <> g' y}, et la contradiction vient du fait qu'on peut aussi montrer \mintinline{coq}{g y = g' y} avec l'extentionnalité propositionnelle.

\item Peut-on se passer du tiers exclu ?

\textbf{Indications :} On peut en effet se passer du tiers exclu, mais pas de l'axiome d'extentionnalité propositionnelle.
On pourra poser les définitions suivantes :
\begin{minted}{coq}
set (g := fun (y: B) => True).
set (g' := fun (y: B) => exists x, f x = y).
\end{minted}

La tactique \(\mintinline{coq}{replace }~e\mintinline{coq}{ with }~e'\) permet de remplacer dans le but les occurrences de \(e\) par \(e'\) en ajoutant le but supplémentaire \(e' = e\). On peut s'en servir pour remplacer le but \mintinline{coq}{exists x : A, f x = y} par \mintinline{coq}{True}, ce qui nous ramènera à prouver

\mintinline{coq}{True = (exists x : A, f x = y)}

qui est convertible en \mintinline{coq}{g x = g' y} (on peut utiliser la tactique \mintinline{coq}{change} pour passer d'un but convertible à un autre).
\end{enumerate}

\section*{Exercice 2 : Lemme d'affaiblissement}

On se donne un ensemble de variables.

\begin{minted}{coq}
Parameter var : Set.
\end{minted}

\begin{enumerate}
\item Définir un inductif \mintinline{coq}{lambda} pour représenter les termes du \(\lambda\)-calcul (on demande trois constructeurs, pour les variables, les abstractions et les applications).

\textbf{Indication :} voici une définition possible pour le premier constructeur
\begin{minted}{coq}
Inductive lambda :=
| Var (x: var)
| ...
\end{minted}
ce qui est un raccourci d'écriture pour
\begin{minted}{coq}
Inductive lambda :=
| Var : forall x: var, lambda (* ou, de manière équivalente, var -> lambda *)
| ...
\end{minted}
\item Définir un inductif \mintinline{coq}{type} pour représenter les types (on demande seulement deux constructeurs : les types atomiques, qui seront nommés par des variables, et l'implication).

\textbf{Indication :} lorsque plusieurs paramètres ont le même type, on peut les écrire sous la même annotation de type
\begin{minted}{coq}
Inductive type :=
| ...
| Arrow (t u: type).
\end{minted}
\item On représentera un contexte par une valeur de type \mintinline{coq}{list (var * type)}. Définir un inductif \mintinline{coq}{derivation} pour représenter les dérivations de typage en logique naturelle: cet inductif aura trois indices, pour le contexte, le \(\lambda\)-terme et son type. On aura un constructeur pour chacune des trois règles suivantes: la règle axiome et les règles d'introduction et d'élimination de l'implication.

\textbf{Indications :} on pourra importer le module \mintinline{coq}{List} de la bibliothèque standard avec la commande vernaculaire \mintinline{coq}{Require Import List.}, et utiliser le prédicat d'appartenance  \mintinline{coq}{In} qui y est défini pour la règle d'axiome.
\begin{minted}{coq}
Inductive derivation: context -> lambda -> type -> Prop :=
| Ax: forall Gamma, forall x t, In (x, t) Gamma -> derivation Gamma (Var x) t
| ...
\end{minted}

\item Montrer le lemme d'affaiblissement sur ce fragment : pour tous \(\Gamma\), \(x\) et \(t\), si \(\Gamma \vdash x : t\), alors pour tout \(\Gamma'\), si \(\Gamma \subseteq \Gamma'\), alors \(\Gamma' \vdash x : t\).

\textbf{Indication :} on pourra utiliser \mintinline{coq}{incl} du module \mintinline{coq}{List} pour exprimer \(\Gamma \subseteq \Gamma'\).
Pour voir les théorèmes déjà définis pour \mintinline{coq}{In} et \mintinline{coq}{incl}, on pourra utiliser les commandes \mintinline{coq}{Search In} et \mintinline{coq}{Search incl.} 

\item On pourra itérativement élargir le fragment aux autres règles de la logique naturelle.
\end{enumerate}

\section*{Exercice 3 : Diagonale d'une matrice carrée}

On se donne le type des listes de longueur fixée.
\begin{minted}{coq}
Inductive vect {A}: nat -> Type := 
| nil: vect O
| cons: forall {n}, A -> vect n -> vect (S n).
\end{minted}

Définir une fonction \mintinline{coq}{diag: forall A n, @vect (@vect A n) n -> @vect A n} qui extrait la diagonale d'une matrice carrée.

\textbf{Indications :} on commencera par définir les fonctions auxiliaires suivantes.

\begin{itemize}
\item \mintinline{coq}{head : forall {A} {n}, @vect A (S n) -> A}
\item \mintinline{coq}{tail : forall {A} {n}, @vect A (S n) -> @vect A n}
\item \mintinline{coq}{map : forall {A} {B} {n} (f: A -> B), @vect A n -> @vect B n}
\end{itemize}

En observant \mintinline{coq}{Print head}, \mintinline{coq}{Print tail} et \mintinline{coq}{Print map}, comprendre comment Coq/Rocq a pu éliminer les cas de filtrage impossibles.

On fera d'abord un filtrage sur \mintinline{coq}{n} et on utilisera le principe de coupure commutative (ou \emph{convoy pattern}):
lorsqu'on filtre sur une valeur \mintinline{coq}{x} et que \mintinline{coq}{x} et/ou l'un des indices apparaissant dans le type de \mintinline{coq}{x}
apparaît également dans le type d'autres variables du contexte, on peut \(\eta\)-expanser ces variables autour du \mintinline{coq}{match} pour que leur type soit réécrit en fonction des branches.
Par exemple, si on a un contexte avec \mintinline{coq}{n: nat} et \mintinline{coq}{mat: @vect (@vect A (S n)) (S n)}, on peut abstraire
\mintinline{coq}{mat} dans le \mintinline{coq}{match} pour que les occurrences de \mintinline{coq}{n} dans le type de \mintinline{coq}{map} soient remplacées par la valeur prise par \mintinline{coq}{n: nat} dans chacune des branches.
\begin{minted}{coq}
  match n as n0 return forall mat: @vect (@vect A n0) n0, ... with
  | O => fun mat: @vect (@vect A O) O => ...
  | S m => fun mat: @vect (@vect A (S m)) (S m) => ...
  end mat.
\end{minted}
Noter que le prédicat de retour, \mintinline{coq}{as n0 return forall mat: @vect (@vect A n0) n0, ...}, peut le plus souvent être inféré par Coq/Rocq.

En particulier, si on a dans le contexte \mintinline{coq}{mat: @vect (@vect A (S m)) (S m)}, abstraire le témoin d'égalité \mintinline{coq}{eq_refl : n = n} permet de récupérer dans la branche \mintinline{coq}{cons} un témoin d'égalité \mintinline{coq}{m = m'}, où \mintinline{coq}{m'} est le paramètre de longueur porté par \mintinline{coq}{cons}.
\begin{minted}{coq}
match mat with
| @cons _ m' hd tl => fun e : m = m' => ...
end eq_refl
\end{minted}

La même technique permet également d'utiliser un tel témoin d'égalité pour réécrire \mintinline{coq}{m'} en \mintinline{coq}{m} dans le type de \mintinline{coq}{tl : @vect (@vect A m) m'}.

\begin{minted}{coq}
match e in _ = m0 return forall tl : @vect (@vect A (S m)) m0, @vect A (S m) with
| eq_refl => fun (tl : @vect (@vect A (S m)) m) => ...
end (tl : @vect (@vect A (S m)) m')
\end{minted}

Ici encore, les annotations de type sur \mintinline{coq}{tl} et le prédicat de retour sont superflus.        
\end{document}
%%% Local Variables:
%%% coding: utf-8
%%% mode: latex
%%% TeX-engine: luatex
%%% TeX-command-extra-options: "-shell-escape"
%%% End:
