\documentclass{article}
\usepackage{fontspec,mathpartir,amsmath}
\usepackage[a4paper]{geometry}
\usepackage[francais]{babel}
\DeclareMathOperator{\efq}{efq}
\DeclareMathOperator{\case}{case}
\DeclareMathOperator{\of}{of}
\DeclareMathOperator{\dest}{dest}
\DeclareMathOperator{\as}{as}
\DeclareMathOperator{\cin}{in}
\begin{document}
\title{Déduction naturelle}
\author{Preuves assistées par ordinateur -- TD 1 -- 24 janvier 2025}
\date{Année scolaire 2024--2025}
\maketitle

\section*{Exercice 1}

On rappelle les règles d'inférence de la déduction naturelle intuitionniste (sans quantificateur).

\begin{mathpar}
\inferrule{~}{\Gamma_1, a: A, \Gamma_2 \vdash a : A} \quad \quad
\inferrule{~}{\Gamma \vdash () : \top} \quad \quad
\inferrule{\Gamma \vdash p : \bot}{\Gamma \vdash \efq p : A} \\
\inferrule{\Gamma \vdash p : A \\ \Gamma \vdash q : B}{\Gamma \vdash (p, q) : A \wedge B} \quad \quad
\inferrule{\Gamma \vdash p : A \wedge B}{\Gamma \vdash \pi_1(p) : A} \quad \quad
\inferrule{\Gamma \vdash p : A \wedge B}{\Gamma \vdash \pi_2(p) : B} \\
\inferrule{\Gamma \vdash p : A}{\Gamma \vdash \iota_1(p) : A \vee B} \quad \quad
\inferrule{\Gamma \vdash p : B}{\Gamma \vdash \iota_2(p) : A \vee B} \quad \quad
\inferrule{\Gamma \vdash p : A \vee B \\ \Gamma, a : A \vdash q : C \\ \Gamma, b : B \vdash r : C}{\Gamma \vdash \case p \of \, [\iota_1(a) \rightarrow q \vert \iota_2(b) \rightarrow r] : C}\\
\inferrule{\Gamma, a : A \vdash p : B}{\Gamma \vdash \lambda a . p : A \Rightarrow B} \quad \quad
\inferrule{\Gamma \vdash p : A \Rightarrow B \\ \Gamma \vdash q : A}{\Gamma \vdash p \, q : B} \\
\end{mathpar}

\begin{enumerate}
\item Montrer que l'implication est réflexive (\(A \Rightarrow A\))
  et transitive (\((A \Rightarrow B) \wedge (B \Rightarrow C) \Rightarrow (A \Rightarrow C)\)).
  Commenter les programmes obtenus.
\item Montrer l'isomorphisme de Curry
  \[(A \wedge B \Rightarrow C) \Leftrightarrow (A \Rightarrow B \Rightarrow C)\]
  (on note \(A \Leftrightarrow B\) pour \((A \Rightarrow B) \wedge (B \Rightarrow A)\)).

  Quelle conséquence peut-on en déduire en programmation ?
\item Montrer la distributivité de \(\wedge\) et \(\vee\) avec l'implication.
  \[(A \Rightarrow (B \wedge C)) \Leftrightarrow ((A \Rightarrow B) \wedge (A \Rightarrow C))\]
  \[((A \vee B) \Rightarrow C) \Leftrightarrow ((A \Rightarrow C) \wedge (B \Rightarrow C))\]
  Quelles conséquences peut-on en déduire en programmation ?
\item Montrer que \(\top\) est neutre pour \(\wedge\) et absorbant pour \(\vee\).

  Montrer que \(\bot\) est absorbant pour \(\wedge\) et neutre pour \(\vee\).
\end{enumerate}

\section*{Exercice 2}

On rappelle les règles d'inférence de la déduction naturelle intuitionniste pour les quantificateurs.

\begin{mathpar}
\inferrule{\Gamma \vdash p : A \\ \text{\(x\) n'est pas libre dans \(\Gamma\)}}{\Gamma \vdash \lambda x . p : \forall x . A} \quad \quad
\inferrule{\Gamma \vdash p : \forall x \, A}{\Gamma \vdash p \, t : A[t/x]} \\
\inferrule{\Gamma \vdash p : A[t/x]}{\Gamma \vdash (t, p) : \exists x \, A} \quad \quad
\inferrule{\Gamma \vdash p : \exists x \, A \\ \Gamma, a : A \vdash q : C \\ \text{\(x\) n'est pas libre dans \(\Gamma\)}}{\Gamma \vdash \dest p \as \, (x, a) \cin q : C}
\end{mathpar}
\begin{enumerate}
\item Montrer la distributivité de \(\wedge\) avec la quantification universelle
et la distributivité de \(\vee\) avec la quantification existentielle.
\[
\forall x (A \wedge B) \Leftrightarrow (\forall x \, A) \wedge (\forall x \, B)
\]
\[
\exists x (A \vee B) \Leftrightarrow (\exists x \, A) \vee (\exists x \, B)
\]
\item Qu'en est-il de la distributivité de \(\wedge\) avec la quantification existentielle
et de la distributivité de \(\vee\) avec la quantification universelle ?
\item Montrer les équivalences \(\forall x \, \forall y \, A \Rightarrow \forall y \, \forall x \, A\)
et \(\exists x \, \exists y \, A \Rightarrow \exists y \, \exists x \, A\).
\item Montrer l'implication \(\exists y \, \forall x \, A \Rightarrow \forall x \, \exists y \, A\).
  Justifier informellement qu'on ne peut pas montrer l'implication réciproque.
\end{enumerate}

\section*{Exercice 3}

On note \(\neg A\) l'implication \(A \Rightarrow \bot\).

\begin{enumerate}
\item Montrer que le tiers exclu (\(A \vee \neg A\)) n'est pas prouvable en logique intuitionniste.
\item Montrer que \(A \Rightarrow \neg \neg A\) est prouvable en logique intuitionniste.

  Montrer que la réciproque est équivalente au tiers exclu.
\item Montrer que \(\neg (A \vee B) \Leftrightarrow (\neg A) \wedge (\neg B)\)
et montrer que \((\neg A) \vee (\neg B) \Rightarrow \neg (A \wedge B)\).
\item Montrer que \(\neg (\exists x \, A) \Leftrightarrow \forall x (\neg A)\)
et montrer que \(\exists x (\neg A) \Rightarrow \neg (\forall x \, A)\).
\item Montrer que la loi de Pierce ci-dessous est équivalente au tiers exclu.
  \[((A \Rightarrow B) \Rightarrow A) \Rightarrow A\]
\item On suppose que \(\Gamma\) contient \(\text{call/cc} : ((A \Rightarrow B) \Rightarrow A) \Rightarrow A\).

Montrer que, dans le contexte \(\Gamma\), on a \(\neg (A \wedge B) \Rightarrow (\neg A) \vee (\neg B)\)
et \(\neg (\forall x \, A) \Rightarrow \exists x (\neg A)\).

  Quelles conséquences peut-on en déduire en programmation ?
\end{enumerate}

\end{document}
