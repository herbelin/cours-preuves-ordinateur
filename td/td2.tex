\documentclass{article}
\usepackage{fontspec,mathpartir,amsmath}
\usepackage[a4paper]{geometry}
\usepackage[francais]{babel}
\DeclareMathOperator{\efq}{efq}
\DeclareMathOperator{\case}{case}
\DeclareMathOperator{\of}{of}
\DeclareMathOperator{\dest}{dest}
\DeclareMathOperator{\as}{as}
\DeclareMathOperator{\cin}{in}
\begin{document}
\title{Logique classique et double négation}
\author{Preuves assistées par ordinateur -- TD 2 -- 14 mars 2025}
\date{Année scolaire 2024--2025}
\maketitle

\section*{Exercice 1 : double négation de l'implication}
Montrer que les énoncés suivants sont tous trois équivalents en logique intuitionniste.

\begin{itemize}
\item \(\neg\neg (A \Rightarrow B)\)
\item \(\neg \neg A \Rightarrow \neg \neg B\)
\item \(A \Rightarrow \neg \neg B\)
\end{itemize}

\section*{Exercice 2 : variantes du tiers exclu}

Montrer que toutes ces lois (pour toutes propositions \(P\), \(Q\), \(R\)) sont équivalentes en logique intuitionniste.

\begin{itemize}
\item \(P \vee \neg P\) (tiers exclu)
\item \(\neg \neg P \Rightarrow P\) (élimination de la double négation)
\item \((\neg P \Rightarrow P) \Rightarrow P\) (\emph{mirabile consequentia}, loi de Clavius)
\item \(((P \Rightarrow Q) \Rightarrow P) \Rightarrow P\) (loi de Peirce)
\item \((\neg (P \Rightarrow Q)) \Rightarrow P \wedge \neg Q\) (principe du contre-exemple)
\item \(P \vee (P \Rightarrow Q)\) (formule de Tarski)
\item \((P \Rightarrow Q) \vee (Q \Rightarrow R)\) (principe de linéarité)
\end{itemize}

Montrer que le tiers exclu faible (\(\neg P \vee \neg \neg P\)) est équivalent à la loi de de Morgan (\(\neg (P \wedge Q) \Rightarrow \neg P \vee \neg Q\)).

\section*{Exercice 3 : traduction négative}

Soit \(\Phi\) une formule propositionnelle en les variables \(P_1, \dots, P_n\).

\begin{enumerate}
\item Supposons \(\neg \neg \Phi\) prouvable en logique intuitionniste. Montrer que \(\Phi\) est prouvable classiquement.
\item Supposons \(\Phi\) prouvable classiquement. Montrer que \((P_1 \vee \neg P_1) \Rightarrow \dots \Rightarrow (P_n \vee \neg P_n) \Rightarrow \Phi\) est prouvable en logique intuitionniste. En déduire que \(\neg \neg \Phi\) est prouvable en logique intuitionniste (théorème de Glivenko, 1929).
\item En déduire que si la logique intuitionniste est cohérente, alors la logique classique est cohérente.
\end{enumerate}

\end{document}
